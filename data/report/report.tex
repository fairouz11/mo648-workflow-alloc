\documentclass[a4paper,10pt, draft]{article}

% escreve textos gerados em portugues
\usepackage[brazilian]{babel}
% aceita unicode
\usepackage[utf8]{inputenc}


\title{Comparação de algoritmos de escalonamento de workflows para redes híbridas}
\author{Aloisio Vilas-Boas\and Andre Petris Esteve\and Atilio Gomes Luiz}
\date{\today}

%% \pdfinfo{%
%%   /Title    ()
%%   /Author   ()
%%   /Creator  ()
%%   /Producer ()
%%   /Subject  ()
%%   /Keywords ()
%% }

\begin{document}

\maketitle

\begin{center}
--------------------------------------------------------------------------------------------\\
Universidade Estadual de Campinas\\
Instituto de Computação\\
MO648 - Projeto de Redes Multimídia\\
Prof. Dr. Nelson Luis Saldanha da Fonseca\\
--------------------------------------------------------------------------------------------\\
\end{center}

\section{Introdução}

(Escrever algo aqui)

\section{Motivação}

(Escrever algo aqui)

\section{Descrição do contexto}

Um usuário deseja executar um workflow de tarefas representado por um \emph{grafo acíclico 
direcionado} (DAG). Cada tarefa é representada por um nó no DAG. Se uma tarefa depende de
resultados de outra tarefa, então há um arco ligando os nós das respectivas tarefas no DAG.
Um nó possui um peso relacionado ao número de intruções necessárias para que a tarefa seja executada.
Já um arco possui um peso relacionado ao tamanho dos dados que devem ser transferidos entre tarefas
dependentes.

O usuário tem disponível uma infraestrutura de nuvem híbrida para execução do 
seu workflow. Essa infra-estrutura é composta por diferentes nuvens: (a) uma nuvem 
privada, onde execuções de tarefas não acarretam custos monetários ao usuário; e (b) 
nuvens públicas nas quais o usuário paga pelo tempo de processamento utilizado.

Este usuário tem um requisito de qualidade de serviço relacionado 
ao tempo de execução do seu workflow, que chamamos de deadline. Dessa forma, o tempo 
de execução do workflow deve ser menor que o deadline.

O objetivo do usuário é realizar o escalonamento de seu DAG nos recursos da 
nuvem híbridas dentro do deadline estipulado minimizando o custo monetário.

\section{Objetivo do projeto}

Este projeto teve como objetivo comparar diferentes algoritmos de escalonamento 
para redes híbidas. Foram implementados quarto algoritmos: \emph{escalonador aleatório}, \emph{round robin},
\emph{cost-driven scheduling using partial critial paths} e \emph{algoritmo de otimização por nuvem de partículas}.
Os algoritmos são descritos na próxima seção.

\section{Descrição dos Algorítmos}

\subsection{Algoritmo Aleatório}

O algoritmo baseia-se na alocação aleatória de recursos para o processamento de tarefas do workflow.

Este escalonador realiza os seguintes passos:

\begin{enumerate}

    \item Realize a ordenação topológica do grafo de workflows.

    \item Enquanto o grafo for não vazio, escolha um nó sem dependentes, ou seja, uma tarefa que não possua dependências 
(ou cujas dependências já foram executadas).

    \item Aleatoriamente selecione um recurso da nuvem e atribua a execução da tarefa selecionada ao recurso.

    \item Remova o nó (tarefa) do grafo (workflow) e volte ao item 2 caso haja ainda tarefas a serem alocadas.

\end{enumerate}

O escalonador aleatório foi implementado apenas para servir como um limitante inferior para a análise.
Não esperamos que nenhum dos demais escalonadores tenha uma performance inferior ao deste algoritmo.

\subsection{Algoritmo Round Robin}

O algoritmo baseia-se na alocação seqüêncial de recursos para o processamento de tarefas do workflow.

Este escalonador realiza os seguintes passos:

\begin{enumerate}

    \item Realize a ordenação topológica do grafo de workflows.

    \item A cada recurso da nuvem atribua um identificador único entre 0 e N-1, onde N é o número de recursos.

    \item Defina o contador 'i' como 0.

    \item Enquanto o grafo for não vazio, escolha um nó sem dependentes, ou seja, uma tarefa que não possua dependências 
(ou cujas dependências já foram executadas).

    \item Selecione o recurso da nuvem cujo identificador é (i mod N).

    \item Atribua a execução da tarefa selecionada ao recurso obtido no item 5.

    \item Incremente o contador 'i'.

    \item Remova o nó (tarefa) do grafo (workflow) e volte ao item 4 caso haja ainda tarefas a serem alocadas.

\end{enumerate}

Este escalonador serve como base para as comparações. Espera-se que os demais algoritmos (com exceção do escalonador
aleatório) apresentem melhor performance.

\subsection{Cost-driven Scheduling Using Partial Critical Paths}

(Descrever algorítmo)

\subsection{Algoritmo de Otimização por Nuvem de Partículas}

A solução para o problema de escalonamento, apresentada no artigo [1], é baseada numa heurística 
denominada \emph{Particle Swarm Optimization} (PSO) ou \emph{Otimização por Nuvem de Partículas}. A heurística de 
escalonamento tem como objetivo otimizar 
o custo do mapeamento tarefa-máquina baseada na solução dada pelo PSO. O processo de otimização utiliza 
dois algoritmos principais: (a) A heurística de escalonamento listada no algoritmo 1, e (b) Os passos do PSO 
para otimização do mapeamento tarefa-máquina, listado no algoritmo 2. Antes de apresentar os algoritmos, vamos 
dar uma breve descrição do algoritmo PSO.

\subsubsection{Particle Swarm Optimization - PSO}

Particle Swarm Optimization(PSO) ou Otimização por Nuvem de Partículas é um algoritmo originalmente 
desenvolvido por Eberhart e Kennedy[3]. O PSO é inspirado na simulação do comportamento social de um 
bando de pássaros em revoada, com movimento localmente aleatório, mas globalmente determinado. No PSO, 
assim como em outros algoritmos, existe uma população de indivíduos, chamados de nuvem (ou enxame) de 
partículas, que em vez de utilizar operadores genéticos, evoluem através de cooperação e competição, entre si, 
por diversas gerações. As partículas se beneficiam de sua própria experiência e da experiência de outros membros 
do enxame durante a busca de um melhor fitness (alvo: comida, local para repouso, proteção de predadores, etc).

\subsubsection{Estrutura do PSO}

O PSO é muito similar a algumas técnicas da computação evolucionária, como os algoritmos genéticos, onde o 
sistema é inicializado com uma população de soluções randômicas. Entretanto, difere dessas outras técnicas 
populacionais, pois nenhum operador inspirado pelos procedimentos de DNA, é aplicado na população, para obter 
uma nova geração de indivíduos. Em vez de mutação, são utilizadas partículas, movendo-se em um espaço de busca 
n-dimensional, sendo cada uma delas um solução potencial para o problema. Cada partícula possui também uma 
velocidade que é inicializada aleatoriamente, para que possa percorrer o espaço de soluções do problema.

A i-ésima partícula é representada por:

Xi = (Xi1, Xi2, …, Xid)

A melhor solução prévia, ou seja, a posição que dá o melhor valor de aptidão da i-ésima partícula, é registrada e representada por:

Pi = (Pi1, Pi2, …, Pid)

A taxa da mudança de posição, que é chamada de velocidade, para a partícula i, é representada por:

Vi = (Vi1, Vi2, …, Vid)

As partículas são manipuladas pelas seguintes equações:

(1) v(i, k+1) = w*v(i, k) + c1*rand1 * (pbest(i) – x(i, k)) + c2*rand2*(gbest – x(i, k))


(2) x(i, k+1) = x(i, k) + v(i, k+1)


Onde:\\

v(i, k) – velocidade da partícula i na k-ésima iteração\\
v(i, k+1) – velocidade da partícula i na (k+1)-ésima iteração\\
w – fator de inércia que determina a diversificação ou a intensificação das partículas\\
c(j) – coeficientes de aceleração; j=1, 2\\
rand(i) – números randômicos entre 0 e 1; i =1, 2\\
x(k, i) – posição atual da partícula i na k-ésima iteração\\
pbest(i) – melhor posição da partícula i\\
gbest – posição da melhor partícula da população\\
x(i, k+1) – posição da partícula i na (k+1)-ésima iteração\\

A Equação (1) é usada para calcular a nova velocidade da partícula, de acordo com sua velocidade 
anterior e as distâncias entre sua posição atual, sua melhor posição, e a melhor posição do grupo. 
A partir daí a partícula muda para uma nova posição de acordo com a Equação (5). O desempenho de cada 
partícula é medido de acordo com uma função de aptidão pré-definida que é relacionada ao problema a ser 
resolvido. No nosso caso, essa função de aptidão tenta minimizar os custos das alocações dos 
recursos feitas em cada partícula.\\

Cada partícula mantém o rastro de suas coordenadas no espaço-problema, que estão associadas à 
melhor solução (fitness) que ela tenha encontrado até então. O valor do fitness também é armazenado. 
Esse valor é chamado de pbest. Outro valor que é rastreado é o melhor entre todos os valores, e sua posição, 
obtido por qualquer partícula na população. Esse valor é chamado de gbest.\\

O conceito do PSO consiste a cada iteração, em mudança de velocidade, as partículas 
alcancem suas posições de pbest e gbest. O algoritmo PSO, tal como descrito pelos 
autores do artigo, pode ser visto na Figura 1.

\section{O simulador}

Para realizar a comparação dos algoritmos, foi necessário utilziar um simulador de núvens. O simulador é
responsável por receber e tratar os pedidos de alocação de tarefas, bem como os pedidos de transmissão dos
resultados de uma tarefa executada.

Foi implementado um simulador que atendesse aos requisitos descritos anteriormente. A implementação baseou-se
em um, já consolidado, simulador de núvens chamado \emph{cloudsim}\footnote{http://www.cloudbus.org/cloudsim/}.

As alterações feitas e decisões de projeto tomadas são descrita com detalhes abaixo.

\subsection{Arquitetura}

A fim de que fosse possível testar um número variado de algoritmos e que a implementação do escalonador
não estivesse altamente acoplada ao funcionamento do simulador, decidimos adicionar uma camada de abstração
ào projeto.

Essa camada de abstração provê funcionalidades básicas ao implementador do escalonador, sem revelar como o
simulador as implementa. Desta forma, é possível alterar o simulador subjacente sem alterar o código dos escalonadores,
bem como é possível desenvolver um escalonador sem conhecer qual será o simulador utilizado.

Do ponto de vista da implementação, a camada de abstração é um conjunto de interfaces que define os contratos
entre os escalonadores e o simulador.
Ela é genérica o suficiente para permitir que diferentes escalonadores sejam suportados por diferentes simuladores.

O principal motivo dessa decisão de projeto foi permitir o desenvolvimento em paralelo da implementação dos escalonadores
e do simulador.

Portanto, a arquitetura do projeto é como vista abaixo.

\begin{center}
----------------------------------------------------------\\
                ESCALONADORES                 \\
----------------------------------------------------------\\
             CAMADA DE ABSTRAÇÃO              \\
----------------------------------------------------------\\
        SIMULADOR (COUDSIM MODIFICADO)        \\
----------------------------------------------------------\\
\end{center}

\subsection{Interface do escalonador}

O escalonador deve implementar uma interface que permite as seguintes
interações entre ele e o simulador:

\begin{enumerate}

  \item Inicialização
  \item Evento: fim de execução de tarefa previamente alocada
  \item Evento: fim de transmissão de dados

\end{enumerate}

Abaixo descrevemos cada possível interação entre o simulador e o escalonador.

Em (1) o escalonador deve preparar o ambiente para iniciar as alocações. Se o escalonador
não se importar com os eventos do simulador, ele pode realizar a alocação de todas as tarefas
neste momento. Nota: ao menos uma tarefa deve ser alocada nesta etapa, pois caso nenhuma tarefa
seja alocada o simulador termina, uma vez que não há o que simular.

Em (2) o simulador avisa ao escalonador que uma tarefa previamente alocada terminou sua execução.
O escalonador pode obter informações sobre o processamento, bem como alocar novas tarefas, se desejar.

Em (3) o simulador avisa ao escalonador que uma transmissão de dados terminou.
O escalonador pode obter informações sobre a transmissão, bem como alocar novas tarefas, se desejar.

\subsection{Interface do simulador}

O simulador deve, por sua vez, implementar uma interface que permite que escalonadores
realizem operações básicas durante a simulação. Essas operações são:

\begin{enumerate}

  \item Obter tarefas (DAG do workflow)
  \item Obter recursos (recursos da rede pública e privada)
  \item Alocar uma tarefa
  \item Alocar uma transmissão

\end{enumerate}

Abaixo descrevemos cada funcionalidade:

Em (1) o simulador provê o DAG que contém as tarefas, suas dependências e os pesos das tarefas (número
de instruções) e os pesos dos arcos (tamanho dos dados a serem transmitidos entre tarefas dependentes).

Em (2) o simulador provê uma lista de recursos da nuvem hibrida. Os recursos da nuvem privada diferem
dos da nuvem publica por não possuirem custo atrelado a sua utilização.

Em (3) o simulador permite que o escalonador aloque uma tarefa para execução em um recurso da nuvem.

Em (4) o simulador permite que o escalonador aloque uma transmissão de dados (resultado de uma tarefa)
entre recursos da rede. A limitação imposta aqui é que uma transmissão só é válida se ela iniciar em um
recurso que executou a tarefa cujos dados estão sendo transmitidos.

\subsection{Alterações do simulador}

Nesta seção descrevemos quais foram os requisitos que não estavam presentes no simulador original
e que foram implementados na nossa versão.

Utilizamos o \emph{cloudsim} release 2.1.1 como base para as modificações.
 
O simulador de núvens \emph{cloudsim} não serviu imediatamente ao nosso propósito. Algumas funcionalidades
não eram implementadas ou eram parcialmente implementadas:

\begin{enumerate}

  \item Simulação da transmissão de resultados entre recursos
  \item Simular fila de transmissões
  \item Considerar dependências de tarefas antes de iniciar sua execução

\end{enumerate}

Abaixo exemplificamos cada requisito.

Considere duas tarefas A e B, sendo que B depende dos resultados de A para executar.
Considere também dois recursos $\alpha$ e $\beta$.
Se A for alocada em $\alpha$ e B em $\beta$, então antes de B iniciar em $\beta$ é necessário
que sejam transmitidos os dados gerados por A de $\alpha$ para $\beta$.

Realizar a simulação da transmissão e considerar o tempo gasto durante o processo é o
requisito descrito em (1).

Tendo em mente o exemplo anterior, suponha que foi alocada uma terceira tarefa, C, em
$\alpha$, antes que A fosse alocada. Contudo, o escalonador já previu a transmissão do
resultado de A para $\beta$. Assim é necessário manter uma fila de transmissões entre
recursos. Desta forma, é possível simular um canal entre dois recursos e garantir que
o canal realize uma transmissão por vez, bem como enfileirar transmissões requisitadas
em um canal em uso. Este é o requisito (2).

O requisito (3) é necessário uma vez que uma tarefa não pode ser executada até que todas
suas dependências tenham sido transmitidas ao recurso que realizará a execução. Desta forma,
caso uma tarefa seja alocada e suas dependências ainda não tenham terminado de ser transmitidas
a tarefa consome processamento do recurso, mas não avança sua execução.

\section{Metodologia}

(Descrever de maneira geral o processo de execução)

\subsection{Grafos usados}

(Quais foram os grafos gerados)

\subsection{Experimentos realizados}

(Descrever os experimentos - tipos de coisas q vamos plotar e analisar)

\subsection{Resultados}

(Resultados, gráficos...)

\section{Conclusão}

(Conclusões)


\end{document}
